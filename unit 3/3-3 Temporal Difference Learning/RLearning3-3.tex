\documentclass[11pt,table]{beamer}
\mode<presentation>
\usepackage{etex}
\usepackage{graphicx}
\usepackage{epstopdf}
\usepackage[english]{babel}
\usepackage{tabularx}
\usepackage{booktabs}
\usepackage{mathrsfs}
\usepackage{multicol}
\usepackage{bm}
\usepackage{subcaption}
\usepackage{wrapfig}
\usepackage{dcolumn}
\usepackage{threeparttable}
\usepackage{booktabs}
\usepackage{bbm}
\usepackage{amsmath,dsfont,listings}
\usepackage{amssymb}
\usepackage{rotating}
\usepackage{multirow}
\usepackage{tcolorbox}
\usepackage[authoryear]{natbib}
\usepackage{circledsteps}
\usepackage{qtree}

\usepackage{tikz}
\usetikzlibrary{arrows,decorations.pathmorphing,backgrounds,fit,positioning,shapes.symbols,chains}
\setbeamertemplate{section in toc}[sections numbered]
\setbeamertemplate{caption}[numbered]

\bibliographystyle{Econometrica}

\setbeamersize{text margin right=3.5mm, text margin left=7.5mm}  % text margin
\setbeamersize{sidebar width left=0cm, sidebar width right=0mm}
\setbeamertemplate{sidebar right}{}
\setbeamertemplate{sidebar left}{}

\definecolor{text-grey}{rgb}{0.45, 0.45, 0.45} % grey text on white background
\definecolor{bg-grey}{rgb}{0.66, 0.65, 0.60} % grey background (for white text)
\definecolor{fu-blue}{RGB}{0, 51, 102} % blue text
\definecolor{fu-green}{RGB}{153, 204, 0} % green text
\definecolor{fu-red}{RGB}{204, 0, 0} % red text (used by \alert)
\definecolor{BrewerBlue}{HTML}{377EB8} % Define Brewer Blue
\definecolor{BrewerRed}{HTML}{E41A1C}  % Define Brewer Red

\setbeamertemplate{frametitle}{%
    \vskip-30pt \color{text-grey}\large%
    \begin{minipage}[b][23pt]{\textwidth}%
    \flushleft\insertframetitle%
    \end{minipage}%
}

\setbeamertemplate{navigation symbols}{} 

%%% begin title page
\setbeamertemplate{title page}{
\vskip2pt\hfill
\vskip19pt\hskip3pt

% set the title and the author
\vskip4pt
\parbox[top][1.35cm][c]{11cm}{\LARGE\color{text-grey} \textcolor{red1}{RL}earning:\\[1ex] \inserttitle \\[1ex] \small \quad \\[3ex]}
\vskip17pt
\parbox[top][1.35cm][c]{11cm}{\small Unit 3-3: \insertsubtitle \\[2ex] \insertauthor \\[1ex]}
}
%%% end title page

%%% colors
\usecolortheme{lily}
\setbeamercolor*{normal text}{fg=black,bg=white}
\setbeamercolor*{alerted text}{fg=fu-red}
\setbeamercolor*{example text}{fg=fu-green}
\setbeamercolor*{structure}{fg=fu-blue}

\setbeamercolor*{block title}{fg=white,bg=black!50}
\setbeamercolor*{block title alerted}{fg=white,bg=black!50}
\setbeamercolor*{block title example}{fg=white,bg=black!50}

\setbeamercolor*{block body}{bg=black!10}
\setbeamercolor*{block body alerted}{bg=black!10}
\setbeamercolor*{block body example}{bg=black!10}

\setbeamercolor{bibliography entry author}{fg=fu-blue}
\setbeamercolor{bibliography entry journal}{fg=text-grey}
\setbeamercolor{item}{fg=fu-blue}
\setbeamercolor{navigation symbols}{fg=text-grey,bg=bg-grey}
%%% end colors

%%% headline
\setbeamertemplate{headline}{
\vskip30pt
}
%%% end headline

%%% footline
\newcommand{\footlinetext}{
%\insertshortinstitute, \insertshorttitle, \insertshortdate
}
\setbeamertemplate{footline}{
\vskip2pt
\hfill \raisebox{-1pt}{\usebeamertemplate***{navigation symbols}}
\hfill \insertframenumber\hspace{10pt}
\vskip4pt
}
%%% end footline

%%% settings for listings package
\lstset{extendedchars=true, showstringspaces=false, basicstyle=\footnotesize\sffamily, tabsize=2, breaklines=true, breakindent=10pt, frame=l, columns=fullflexible}
\lstset{language=Java} % this sets the syntax highlighting
\lstset{mathescape=true} % this switches on $...$ substitution in code
% enables UTF-8 in source code:
\lstset{literate={ä}{{\"a}}1 {ö}{{\"o}}1 {ü}{{\"u}}1 {Ä}{{\"A}}1 {Ö}{{\"O}}1 {Ü}{{\"U}}1 {ß}{\ss}1}
%%% end listings

\usepackage{concmath}
\usepackage{xcolor}
\definecolor{red1}{RGB}{206, 17, 38}
\definecolor{blue1}{RGB}{16, 118, 208}
\definecolor{gray1}{RGB}{117, 115, 115}
\usepackage{hyperref}


\newtheorem{proposition}{Proposition}
\newtheorem{assumption}{Definition}

\title[]{Short guides to reinforcement learning}
\subtitle[]{Temporal Difference Learning}
\author[D. Rostam-Afschar]{\textcolor{gray1}{Davud Rostam-Afschar (Uni Mannheim)}}
\date[]{\today}
\subject{Econometrics}
\renewcommand{\footlinetext}{\insertshortinstitute, \insertshorttitle, \insertshortdate}
\hypersetup{
    bookmarks=false,
    unicode=false,
    pdftoolbar=false,
    pdffitwindow=true,
    pdftitle={Reinforcement Learning for Business, Economics, and Social Sciences: \insertsubtitle},
    pdfauthor={Davud Rostam-Afschar},
    pdfsubject={Reinforcement Learning},
    pdfkeywords={reinforcement learning, Temporal Difference Learning},
    pdfnewwindow=true,
}
\def\sym#1{\ifmmode^{#1}\else\(^{#1}\)\fi}

\begin{document}

\begin{frame}[plain]
  \titlepage
\end{frame}

% --------------------------------------------------- Slide --
%\begin{frame}
	%\frametitle{Content}
	%\tableofcontents[]
%\end{frame}

\section{Temporal Difference Learning}
{
\setbeamercolor{background canvas}{bg=BrewerBlue}
\begin{frame}
\centering
\Huge
\textcolor{white}{How can we learn by sampling from each step?}
\thispagestyle{empty}
\end{frame}
}



\begin{frame}{RL Algorithms}
\only<1>{
Dynamic Programming Backup
\begin{figure}
	\centering
		\includegraphics[width=0.60\textwidth]{figures/dynamic_programming_backup.png}
	\label{fig:DP}
\end{figure}
}
\only<2>{
Monte Carlo Backup
\begin{figure}
	\centering
		\includegraphics[width=0.60\textwidth]{figures/monte_carlo_backup.png}
	\label{fig:MC}
\end{figure}
}
\only<3>{
Temporal Difference Backup
\begin{figure}
	\centering
		\includegraphics[width=0.60\textwidth]{figures/temporal_difference_backup.png}
	\label{fig:TD}
\end{figure}
}
\medskip
\hfill\footnotesize\textit{Source: David Silver}
\end{frame}


\begin{frame}{Model Free Evaluation}

\begin{itemize}
    \item Given a policy $\pi$ estimate $V^{\pi}(s)$ without any transition\\ or reward model

\item \textbf{Temporal difference} (TD) evaluation
$$
\begin{aligned}
& V^{\pi}(s)=E[r \mid s, \pi(s)]+\gamma \sum_{s^{\prime}} \mathbb{P}\left(s^{\prime} \mid s , \pi(s)\right) V^{\pi}\left(s^{\prime}\right) \\
& \textcolor{red1}{\approx r+\gamma V^{\pi}\left(s^{\prime}\right) \quad \text { (one draw approximation) }}
\end{aligned}
$$
\end{itemize}
    
\end{frame}




\begin{frame}{Toy Maze Example}


\begin{columns}[T]
\begin{column}{0.45\textwidth}
\centering
\scalebox{0.6}{

\begin{tikzpicture}[line cap=round,line join=round,x=0.75pt,y=0.75pt,yscale=-1,xscale=1]

%Shape: Rectangle [id:dp08468554766918346] 
\draw  [fill=BrewerBlue] (240.34,110.03) -- (300.07,110.03) -- (300.07,169.77) -- (240.34,169.77) -- cycle ;
\draw[line width=2pt]   (180.53,50) -- (420.11,50) -- (420.11,229.53) -- (180.53,229.53) -- cycle ;
%Straight Lines [id:da4224686848111512] 
\draw[line width=1.5pt]    (240.03,49.53) -- (239.53,229.03) ;
%Straight Lines [id:da6046918343336376] 
\draw[line width=1.5pt]    (300.03,49.53) -- (299.53,229.53) ;
%Straight Lines [id:da9626667234054846] 
\draw[line width=1.5pt]   (360.03,50.03) -- (359.53,229.53) ;
%Straight Lines [id:da267347571646644] 
\draw[line width=1.5pt]    (180.11,110) -- (419.53,110.53) ;
%Straight Lines [id:da6745979138164762] 
\draw[line width=1.5pt]    (180.03,169.53) -- (420.11,170) ;
%Shape: Square [id:dp1839161508205971] 

% Text Node
\draw (159.47,68) node [anchor=north west][inner sep=0.75pt]   [align=left] {{\Large $3$}};
% Text Node
\draw (159.47,130) node [anchor=north west][inner sep=0.75pt]   [align=left] {{\Large $2$}};
% Text Node
\draw (159.47,186) node [anchor=north west][inner sep=0.75pt]   [align=left] {\textbf{{\Large $1$}}};
% Text Node
\draw (201.47,234) node [anchor=north west][inner sep=0.75pt]   [align=left] {{\Large $1$}};
% Text Node
\draw (263.47,234) node [anchor=north west][inner sep=0.75pt]   [align=left] {{\Large $2$}};
% Text Node
\draw (325.47,234) node [anchor=north west][inner sep=0.75pt]   [align=left] {{\Large $3$}};
% Text Node
\draw (381.47,234) node [anchor=north west][inner sep=0.75pt]   [align=left] {{\Large $4$}};
% Text Node
\draw (202.47,70) node [anchor=north west][inner sep=0.75pt]  [font=\huge] [align=left] {{\textbf{r}}};
% Text Node
\draw (264.47,70) node [anchor=north west][inner sep=0.75pt]  [font=\huge] [align=left] {{\huge \textbf{r}}};
% Text Node
\draw (324.47,70) node [anchor=north west][inner sep=0.75pt]  [font=\huge] [align=left] {{\huge \textbf{r}}};
% Text Node
\draw (370.47,70) node [anchor=north west][inner sep=0.75pt]  [font=\huge] [align=left] {$\boldsymbol{+1}$};
% Text Node
\draw (202.47,131) node [anchor=north west][inner sep=0.75pt]  [font=\huge] [align=left] {\textbf{u}};
% Text Node
\draw (322,131) node [anchor=north west][inner sep=0.75pt]  [font=\huge] [align=left] {\textbf{u}};
% Text Node
\draw (370.47,131) node [anchor=north west][inner sep=0.75pt]  [font=\huge] [align=left] {$\boldsymbol{-1}$};
% Text Node
\draw (202.47,188) node [anchor=north west][inner sep=0.75pt]  [font=\huge] [align=left] {\textbf{u}};
% Text Node
\draw (260,186) node [anchor=north west][inner sep=0.75pt]  [font=\huge] [align=left] {\textbf{l}};
% Text Node
\draw (322,186) node [anchor=north west][inner sep=0.75pt]  [font=\huge] [align=left] {\textbf{l}};
% Text Node
\draw (383,186) node [anchor=north west][inner sep=0.75pt]  [font=\huge] [align=left] {\textbf{l}};
\end{tikzpicture}
}
\end{column}
\begin{column}{0.55\textwidth}
Start state: (1,1)\\
Terminal states: (4,2), (4,3)  
\\No discount: $\gamma$=1

\vspace{2mm}
Reward is -0.04 for  non-terminal states

\end{column}
\end{columns}

\vspace{6mm}
\centering
\begin{columns}
\begin{column}{0.4\textwidth}

Four actions:
\begin{itemize}
	\item up (\textbf{u}),
	\item left (\textbf{l}),
	\item right (\textbf{r}),
	\item down (\textbf{d})
\end{itemize}
\end{column}
\begin{column}{0.6\textwidth}
Do not know the transition probabilities

\vspace{3mm}
  \textcolor{red1}{What is the value $V(s)$ of being in state $s$}
\end{column}
\end{columns}
    
\end{frame}


\begin{frame}{Temporal Difference}


\begin{center}

\only<1-2>{
\scalebox{0.8}{
\begin{tikzpicture}[line cap=round,line join=round,x=1pt,y=1pt,yscale=-1,xscale=1]

%Shape: Rectangle [id:dp6447651540962223] 
\draw[line width=2pt]   (200.16,40.44) -- (380.23,40.44) -- (380.23,159.75) -- (200.16,159.75) -- cycle ;
%Straight Lines [id:da6664024825726134] 
\draw[line width=1.5pt]    (260.13,40.44) -- (259.38,160.01) ;
%Straight Lines [id:da787984071032432] 
\draw[line width=1.5pt]    (320.04,40.1) -- (320.38,160.34) ;
%Straight Lines [id:da35411861555941715] 
\draw[line width=1.5pt]    (200.16,99.77) -- (380.23,100.38) ;
%Shape: Square [id:dp6482811179722225] 
\draw  [fill=BrewerBlue  ,fill opacity=1,line width=1.5pt ] (200.42,40.89) -- (259.75,40.89) -- (259.75,100.22) -- (200.42,100.22) -- cycle ;
%Straight Lines [id:da5628793737066431] 
\draw[line width=1.5pt,-latex]    (243.51,59.79) -- (276.78,59.9) ;
%Straight Lines [id:da8088619056184632] 
\draw[line width=1.5pt,latex-]    (243.51,80.12) -- (276.78,80.15) ;
%Straight Lines [id:da023736429487241528] 
\draw[line width=1.5pt,-latex]    (303,70.2) -- (337.78,70.4) ;
%Straight Lines [id:da08303851898715231] 
\draw[line width=1.5pt,-latex]    (290.23,83) -- (290.23,118) ;

% Text Node
\draw (222.8,60) node [anchor=north west][inner sep=0.75pt]  [font=\LARGE] [align=left] {{\Large $\boldsymbol{S_1}$}};
\draw (282,60) node [anchor=north west][inner sep=0.75pt]  [font=\LARGE] [align=left] {{\Large $\boldsymbol{S_2}$}};
\draw (244,48) node [anchor=north west][inner sep=0.75pt]  [font=\LARGE] [align=left] {{\large $\boldsymbol{73}$}};
\draw (244,84) node [anchor=north west][inner sep=0.75pt]  [font=\LARGE] [align=left] {{\large $\boldsymbol{66}$}};
\draw (291,84) node [anchor=north west][inner sep=0.75pt]  [font=\LARGE] [align=left] {{\large $\boldsymbol{81}$}};
\draw (320,57) node [anchor=north west][inner sep=0.75pt]  [font=\LARGE] [align=left] {{\large $\boldsymbol{100}$}};
\end{tikzpicture}
}
}

\only<3>{
\scalebox{0.8}{

\begin{tikzpicture}[line cap=round,line join=round,x=1pt,y=1pt,yscale=-1,xscale=1]

%Shape: Rectangle [id:dp6447651540962223] 

\draw  [fill={BrewerBlue}] (260.87,40.89) -- (320.21,40.89) -- (320.21,100.22) -- (260.87,100.22) -- cycle ;
\draw[line width=2pt]   (200.16,40.44) -- (380.23,40.44) -- (380.23,159.75) -- (200.16,159.75) -- cycle ;
%Straight Lines [id:da6664024825726134] 
\draw[line width=1.5pt]    (260.13,40.44) -- (259.38,160.01) ;
%Straight Lines [id:da787984071032432] 
\draw[line width=1.5pt]    (320.04,40.1) -- (320.38,160.34) ;
%Straight Lines [id:da35411861555941715] 
\draw[line width=1.5pt]    (200.16,99.77) -- (380.23,100.38) ;
%Shape: Square [id:dp6482811179722225] 


%\draw  [fill=bluegray  ,fill opacity=1,line width=1.5pt ] (200.42,40.89) -- (259.75,40.89) -- (259.75,100.22) -- (200.42,100.22) -- cycle ;
%Straight Lines [id:da5628793737066431] 
\draw[line width=1.5pt,-latex]    (243.51,59.79) -- (276.78,59.9) ;
%Straight Lines [id:da8088619056184632] 
\draw[line width=1.5pt,latex-]    (243.51,80.12) -- (276.78,80.15) ;
%Straight Lines [id:da023736429487241528] 
\draw[line width=1.5pt,-latex]    (303,70.2) -- (337.78,70.4) ;
%Straight Lines [id:da08303851898715231] 
\draw[line width=1.5pt,-latex]    (290.23,83) -- (290.23,118) ;

% Text Node
%\draw (222.8,60) node [anchor=north west][inner sep=0.75pt]  [font=\LARGE] [align=left] {{\Large $\boldsymbol{S_1}$}};
\draw (282,60) node [anchor=north west][inner sep=0.75pt]  [font=\LARGE] [align=left] {{\Large $\boldsymbol{S_2}$}};
\draw (244,48) node [anchor=north west][inner sep=0.75pt]  [font=\LARGE] [align=left,color=red] {{\large $\boldsymbol{81.5}$}};
\draw (244,84) node [anchor=north west][inner sep=0.75pt]  [font=\LARGE] [align=left] {{\large $\boldsymbol{66}$}};
\draw (291,84) node [anchor=north west][inner sep=0.75pt]  [font=\LARGE] [align=left] {{\large $\boldsymbol{81}$}};
\draw (320,57) node [anchor=north west][inner sep=0.75pt]  [font=\LARGE] [align=left] {{\large $\boldsymbol{100}$}};
\end{tikzpicture}
}
}

$\gamma=0.9, \alpha=0.5, r=0 \text { for non-terminal states }$

\uncover<2-3>{

\begin{align}
Q(s_{1},right)&=Q(s_{1},right)+\alpha \bigg(r+\gamma \max _{a^{\prime}} Q(s_{2}, a^{\prime})-Q(s_{1},right)\bigg)\nonumber\\
& =73+0.5(0+0.9 \max \{66,81,100\}-73) \nonumber\\
& =73+0.5(17) \nonumber\\
& =81.5\nonumber
\end{align}
}
\end{center}

    
\end{frame}

\section{Temporal Difference Evaluation}
{
\setbeamercolor{background canvas}{bg=BrewerBlue}
\begin{frame}
\centering
\Huge
\textcolor{white}{Temporal Difference Evaluation}
\thispagestyle{empty}
\end{frame}
}


\begin{frame}{Temporal Difference Evaluation}


\begin{itemize}
    \item Approximate value function: $V_{n}^{\pi}(s) \approx r+\gamma V^{\pi}\left(s^{\prime}\right)$
		
		\item \textbf{Incremental update} of sample $\left(\pi, s^{\prime}, s\right)$
   \end{itemize} 

$$
\textcolor{red1}{V_{n}^{\pi}(s) \leftarrow V_{n-1}^{\pi}(s)+\alpha_{n}\left(r+\gamma V_{n-1}^{\pi}\left(s^{\prime}\right)-V_{n-1}^{\pi}(s)\right)}
$$
\end{frame}

\begin{frame}{Exploration vs Exploitation}
Stochastic approximation (Robbins-Monro algorithm)
\begin{itemize}

\item \textbf{Theorem}: If $\alpha_{n}$ is appropriately decreased with number of times a state is visited then $V_{n}^{\pi}(s)$ converges to correct value

\item \textbf{Sufficient conditions} for $\alpha_{n}$ :


\begin{align}
&  \sum_{n} \alpha_{n} \rightarrow \infty \\
& \sum_{n} \alpha^2_{n} < \infty
\end{align}

\item \text Often $\alpha_{n}(s)=1 / n(s)$,\\
where $n(s)=$ \# of times $s$ is visited 
   \end{itemize} 
\end{frame}

\begin{frame}{Temporal Difference (TD) Evaluation}


\begin{tcolorbox}[colframe=black, boxrule=1pt, sharp corners]

\textcolor{red1}{TDevaluation $\left(\pi, V^{\pi}\right)$}


\quad Repeat

\qquad Execute $\pi(s)$

\qquad Observe $s^{\prime}$ and $r$

\qquad Update counts: $n(s) \leftarrow n(s)+1$

\qquad Learning rate: $\alpha \leftarrow 1 / n(s)$

\qquad  Update value: $V^{\pi}(s) \leftarrow V^{\pi}(s)+\alpha\left(r+\gamma V^{\pi}\left(s^{\prime}\right)-V^{\pi}(s)\right)$

\qquad $s \leftarrow s^{\prime}$

\quad Until convergence of $V^{\pi}$

\quad Return $V^{\pi}$


\end{tcolorbox}
    
\end{frame}


\section{Temporal Difference Control}
{
\setbeamercolor{background canvas}{bg=BrewerBlue}
\begin{frame}
\centering
\Huge
\textcolor{white}{Temporal Difference Control}
\thispagestyle{empty}
\end{frame}
}

\begin{frame}{Temporal Difference Control}


\begin{itemize}
    \item Approximate Q-function:

$$
\begin{aligned}
Q^{*}(s, a)= & E[r \mid s, a]+\gamma \sum_{s^{\prime}} \mathbb{P}\left(s^{\prime} \mid s, a\right) \max _{a^{\prime}} Q^{*}\left(s^{\prime}, a^{\prime}\right) \\
& \approx r+\gamma \max _{a^{\prime}} Q^{*}\left(s^{\prime}, a^{\prime}\right)
\end{aligned}
$$

\item \textbf{Incremental update}
\vspace{3mm}
\begin{small}
$\textcolor{red1}{Q_{n}^{*}(s, a) \leftarrow Q_{n-1}^{*}(s, a)+\alpha_{n}\left(r+\gamma \max _{a^{\prime}} Q_{n-1}^{*}\left(s^{\prime}, a^{\prime}\right)-Q_{n-1}^{*}(s, a)\right)}$ 
\end{small}
\end{itemize}
    
\end{frame}



\begin{frame}{Comparison}


    \begin{itemize}
        \item \textbf{Monte Carlo evaluation}:
\begin{itemize}
\item \textcolor{teal}{Unbiased estimate}
\item \textcolor{red1}{High variance}
\item \textcolor{red1}{Needs many trajectories}\\[2ex]
 
\end{itemize} 
\item \textbf{Temporal difference evaluation}:
\begin{itemize}
\item \textcolor{red1}{Biased estimate}
\item \textcolor{teal}{Lower variance}
\item \textcolor{teal}{Needs less trajectories}
 
\end{itemize}
    \end{itemize}
\end{frame}


\begin{frame}[t,allowframebreaks
]\nocite{*}
\frametitle{References}
\footnotesize
\bibliography{bib}
\end{frame}
\section{Takeaways}
{
\setbeamercolor{background canvas}{bg=BrewerBlue}
\begin{frame}
\centering
\Huge
\textcolor{white}{Takeaways}
\thispagestyle{empty}
\end{frame}
}



\begin{frame}{How Does TD Learning Update Value Estimates Step-by-Step?}
\begin{itemize}
    \item No need to know transition probabilities or reward function\\$\rightarrow$ Model free\\[2ex]
    \item Combine immediate reward with estimated value of next state\\$\rightarrow$ Biased value estimation from bootstrapped samples\\[2ex]
    \item Revises estimates after each observed step\\$\rightarrow$ Needs few trajectories\\[2ex]
    \item Lower variance than Monte Carlo at the cost of some bias\\$\rightarrow$ Bias-variance tradeoff\\[2ex]
    \item Often used in algorithms such as Q-learning and SARSA

  \end{itemize}
\end{frame}

\end{document}
